\documentclass[a4paper,11pt]{article}
%\usepackage{macros}


\usepackage[T1]{fontenc}
\usepackage{lmodern}
\usepackage{amsmath}
\usepackage{amsfonts}
\usepackage{amssymb}
\usepackage{amsthm}
\usepackage{mathrsfs}
\usepackage{physics}
\usepackage{graphicx}
\usepackage{caption}
\usepackage{subcaption}
\usepackage{enumitem}
\usepackage{framed,float}
\usepackage{booktabs}
\usepackage{array}
\usepackage{soul}
\usepackage{multirow}
\usepackage{comment}
\usepackage{braket}
\usepackage{todonotes}
\usepackage{youngtab}
\usepackage{tikz}
\usetikzlibrary{calc,arrows,decorations.markings}

\usepackage[left=2.8cm, right=2.8cm, top=3cm, bottom=2.6cm]{geometry}

\usepackage{color}

\theoremstyle{definition}
\newtheorem{definition}{Definition}[section]
\newtheorem{theorem}[definition]{Theorem}
\newtheorem{proposition}[definition]{Proposition}

\begin{document}
\tableofcontents

\section{Introduction}
In this article, 

%%%%%%%%%%%%%%%%
\section{Lattice Ising Model}
I supp

%%%%%%%%%%%%%%%%
\section{Yang-Mills Mass Gap and Confinement}

%%%%%%%%%%%%%%%%
\section{Non-invertible Symmetries and Renormalization}

%%%%%%%%%%%%%%%%
\section{Applying }

%%%%%%%%%%%%%%%%
\section{Conclusion}

%%%%%%%%%
\appendix
\section{Category Theory}
This appendix outlines the essential knowledge of category theory required to understand the main discussion. 
Readers can find more details in \cite{lane1998categories,riehl2017category}.\par
\subsection{Basic Concepts of Category Theory}
\begin{definition}
    A \textit{category} $\mathcal{C}$ consists of:\\
    \begin{itemize}
        \item a collection $C_0$ of \textit{objects};
        \item a collection $C_1$ of \textit{morphisms};
        \item there is an operation $s$, which assigns to each morphism $f$ an object $s(f)$, called its \textit{source} or \textit{domain};
        \item there is an operation $t$, which assigns to each morphism $f$ an object $t(f)$, called its \textit{target} or \textit{codomain};
        \item there is an operation $1$, which assigns to each object $c$ an morphisms $1_c$, called the \textit{identity morphism} on $c$;
        \item there is a composition operation $\circ$, which assigns a pair of morphisms $f$ and $g$, $t(f)=s(g)$, a morphisms $g\circ f$, called their \textit{composite};
        \item such that the following properties are satisfied:
        \begin{itemize}
            \item $s(1_c)=c=t(1_c)$;
            \item $s(g\circ f)=s(f)$ and $t(g \circ f)=t(g)$;
            \item associativity of composition, $(h\circ g)\circ f=h\circ(g\circ f)$;
            \item composition satisfies the unit law, fo $a\xrightarrow{f}b$, $f\circ 1_a=f=1_b\circ f$.
        \end{itemize}
    \end{itemize}
    we may write $c\in \mathcal{C}$ to indicate $c$ is an object of $\mathcal{C}$. And people often write $\mathrm{Hom}(a,b)$, $\mathrm{Hom}_{\mathcal{C}}(a,b)$, $\mathcal{C}(a,b)$ or $\mathcal{C}(a\to b)$
    for the collection of morphisms $f:a\to b$, which is called the hom-set of objects $a$ and $b$.
\end{definition}

\begin{definition}
    An \textit{isomorphism} between two objects $a,b$ in category $\mathcal{C}$ is a morphism $f:a\to b$, 
    and there exsits a morphism $g:b\to a$, such that $f\circ g=1_b$ and $g\circ f=1_a$. 
    $a,b$ are \textit{isomorphic} if there exsits an isomorphism between them. The \textit{isomorphism class} of object $c$ is the set of objects isomorphic to $c$. 
\end{definition}

\begin{definition}
    For categories $\mathcal{C}$ and $\mathcal{D}$, a \textit{functor} $T$ is a morphism with source $\mathcal{C}$ and target $\mathcal{D}$, consists of,
    \begin{itemize}
        \item an \textit{object function} $T_0$, which assigns to each $c\in \mathcal{C}$ an object $T_0c\in \mathcal{D}$;
        \item a \textit{morphism function} $T_1$, which assigns to each morphism $f:c\to c'$ of $\mathcal{C}$ an morphisms $T_1f:T_0c\to T_0c'$ of $\mathcal{D}$;
        \item such that,
        \begin{equation}
            T_1(1_c)=1_{T_0c},\qquad T_1(g\circ f)=T_1g\circ T_1f.
        \end{equation}
    \end{itemize}
    I will write both $T_0$ and $T_1$ as $T$ in this article.
\end{definition}
\begin{definition}
    A natural transformation 
    
\end{definition}

\begin{definition}
    For two given categories $\mathcal{B}$ and $\mathcal{C}$, a new category $\mathcal{B}\times \mathcal{C}$, called the \textit{product} of them, is defined as follows. 
    An object of $\mathcal{B}\times \mathcal{C}$ is a pair $\braket{b,c}$ of $b\in \mathcal{B}$ and $c \in \mathcal{C}$. A morphism $\braket{b,c}\to\braket{b',c'}$ in $\mathcal{B}\times \mathcal{C}$
    is a pair $\braket{f,g}$ of $f:b\to b'$ and $g:c\to c'$. The composition operation in $\mathcal{B}\times \mathcal{C}$ is defined as,
    \begin{equation}    
    \braket{f',g'}\circ\braket{f,g}=\braket{f'\circ f,g'\circ g}
    \end{equation}     
\end{definition}



% \begin{definition}
%     For a functor $T:\mathcal{D}\to \mathcal{C}$ and $c\in \mathcal{C}$, the \textit{universal morphisms} from $c$ to $T$ 
%     is a pair $\braket{r,g}$ consisting of an object $r\in \mathcal{D}$ and a morphism $g:c\to Tr$ of $\mathcal{C}$, 
%     such that to every pair $\braket{s,f}$ with $e\in \mathcal{D}$ and $f:c\to Ts$, there is a unique morphisms $f':r\to s$ of $\mathcal{D}$ with $Tf'\circ u=f$.   
% \end{definition}
\subsection{Unitary Fusion Category}
Main reference for this subsection is \cite{etingof2015tensor}.
\begin{definition}
An \textit{additive category} $\mathcal{C}$  is a category satisfying,
\begin{itemize}
    \item Every hom-set is equipped with a structure of an abelian group such that the the composition operation is biadditive;
    \item There exsits a zero object $0\in \mathcal{C}$ such that $\mathcal{C}(0,0)=0$;
    \item For any objects $c_1,c_2\in \mathcal{C}$, there exsits an object $c\in \mathcal{C}$ and morphisms $p_1:c\to c_1$, $p_2:c\to c_2$, $i_1:c_1\to c$, $i_2:c_2\to c$ 
    such that $p_1\circ i_1=1_{c_1}$, $p_2\circ i_2=1_{c_2}$ and $i_1\circ p_1+i_2\circ p_2=1_c$. The object $c$ is unique up to a unique isomorphism, is denoted by $c_1\oplus c_2$, 
    and called the \textit{direct sum} of $c_1$ and $c_2$. $\oplus$ can be viewed as a functor $\oplus:\mathcal{C}\times \mathcal{C}\to \mathcal{C}$. 
    And we can easily generalize this definition to the case for more than two objects.
\end{itemize}
\end{definition}
\begin{definition}
    If $\mathbb{F}$ is a field, the $\mathbb{F}$-linear category $\mathcal{C}$  is an additive category whose hom-sets are all linear spaces over $\mathbb{F}$, 
    and whose composition operation is $\mathbb{F}$-linear. In physics, the field $\mathbb{F}$ is always chosen to be the complex number $\mathbb{C}$. 
    Unless stated otherwise, all linear spaces in this article are over $\mathbb{C}$. 
\end{definition}
\begin{definition}
    An \textit{idempotent} in category $\mathcal{C}$ is a morphisms $e$ such that $e\circ e=e$. A \textit{splitting} for an idempotent $e:c\to c$ is an triple $(a,r,s)$ 
    where $a\in \mathcal{C},r\in \mathcal{C}(c,a)$, and $s\in \mathcal{C}(a,c)$ such that $s\circ r=e$ and $r\circ s=1_a$. A lineae category is called \textit{idempotent complete} if 
    every idempotent admits a splitting.  
\end{definition}

\begin{definition}
    A linear category $\mathcal{C}$  is called \textit{semisimple} if,
    \begin{itemize}
    \item it admits all finite direct sum;
    \item it is idempotents complete;
    \item there exsits objects $c_i$ labeled by an index set $I$ such that,
    \begin{itemize}
        \item for any $i,j\in I$, we have
        \begin{equation}
            \mathcal{C}(c_i,c_j)\cong \delta_{ij}\mathbb{C},
        \end{equation}
        such objects are called \textit{simple};
        \item for any pair of objects $a,b\in \mathcal{C}$,
        \begin{equation}
            \bigoplus_{i\in I}\mathcal{C}(a,c_i)\otimes \mathcal{C}(c_i,b)\to \mathcal{C}(a,b)
        \end{equation}
        is an isomorphism. Here $\oplus$ and $\otimes$ are direct sum and tensor product for linear spaces respectively. 
    \end{itemize}  
    \end{itemize}
    $\mathcal{C}$ is called \textit{finitely semisimple} is in addition $\mathcal{C}$ has finitely many isomorphism classes of simple objects. 
\end{definition}
I state the following proposition without proof, as it will be used in the text.
\begin{proposition}
    Every object of semisimple linear category is a direct sum of simple objects $c_i$ .
\end{proposition}
\begin{theorem}
    dagger structure
\end{theorem}
\begin{definition}
    $*$-algebra
\end{definition}
\begin{theorem}
    unitary algebra condition
\end{theorem}
\begin{definition}
    unitary algebra
\end{definition}



\bibliographystyle{unsrt}
\bibliography{ref}

\end{document}