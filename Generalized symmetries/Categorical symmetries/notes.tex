\documentclass[a4paper,12pt]{article}
%\usepackage{macros}


\usepackage[T1]{fontenc}
\usepackage{lmodern}
\usepackage{amsmath}
\usepackage{amsfonts}
\usepackage{amssymb}
\usepackage{amsthm}
\usepackage{mathrsfs}
\usepackage{physics}
\usepackage{graphicx}
\usepackage{caption}
\usepackage{subcaption}
\usepackage{enumitem}
\usepackage{framed,float}
\usepackage{booktabs}
\usepackage{array}
\usepackage{soul}
\usepackage{multirow}
\usepackage{comment}
\usepackage{braket}
\usepackage{todonotes}
\usepackage{youngtab}
\usepackage{tikz}
\usetikzlibrary{calc,arrows,decorations.markings}

\usepackage[left=0.5cm, right=0.5cm, top=0.5cm, bottom=1cm]{geometry}

\usepackage{color}

\begin{document}
\section{Gauge theory}
I will not use the language of fibre bundle to construct it strictly, but just simplely write down main ideas.
\subsection{Non-Abelian Gauge Theory}
We have a Lie-algebra-valued one-form $A_\mu^a t^a \mathrm{d}x^\mu\in\mathfrak{g}\times \Lambda^1$, which is called a \textit{connection}. Thus $A_\mu\equiv A_\mu^a t^a\in \mathfrak{g}$. 
We have a Lie-algebra-valued two-form 
\begin{equation}
    F_{\mu \nu}\mathrm{d}x^\mu\wedge \mathrm{d}x^\nu=\partial_\mu A_\nu \mathrm{d}x^\mu\wedge \mathrm{d}x^\nu-\partial_\nu A_\mu \mathrm{d}x^\mu\wedge \mathrm{d}x^\nu-i[A_\mu,A_\nu]\mathrm{d}x^\mu\wedge \mathrm{d}x^\nu
\end{equation}
which is called a \textit{curvature}. 
The Lie-algebra-valued component is $F_{\mu \nu}=\partial_\mu A_\nu-\partial_\nu A_\mu-i[A_\mu,A_\nu]$.\par
In a representation $R$  of the Lie algebra $\mathfrak{g}$, the \textit{covariant derivative} for representation-space-vector-valued scalar (0-form) $\psi$ 
\begin{equation}
    \mathcal{D}_\mu \psi=\partial_\mu \psi-iA_\mu \psi
\end{equation}
The components,
\begin{equation}
    \mathcal{D}_\mu \psi^i=\partial_\mu \psi^i-iA_\mu^a {R(t^a)^i}_j \psi^j,\qquad i,j=1,2,\dots,\dim R
\end{equation}
For a Lie-algebra-valued object $\phi=\phi^a t^a$, the covariant derivative,
\begin{equation}
    \mathcal{D}\phi=\partial \phi-i[A_\mu,\phi].
\end{equation}
$\phi^a$ can be a scalar, one-form or any other. We can get,
\begin{equation}
    [\mathcal{D}_\mu,\mathcal{D}_\nu]=-iF_{\mu \nu}
\end{equation}
I gather some elements of gauge theory above, now we begin the physical part.\par
The action of Yang-Mills,
\begin{equation}
    S_{YM}=\frac{1}{-2g}\int d^4 x\tr(F^{\mu \nu}F_{\mu \nu}).
\end{equation}
The classical e.o.m for this action is,
\begin{equation}
    \mathcal{D}_\mu F^{\mu \nu }=0
\end{equation}
The action has a symmetry, for
\begin{equation}
    \Omega(x)\in G
\end{equation}
The action is invariant under,
\begin{equation}
    A_\mu\to \Omega(x)A_\mu \Omega^{-1}(x)+i \Omega(x)\partial_\mu \Omega^{-1}(x)    
\end{equation} 
leading to,
\begin{equation}
    F^{\mu \nu}\to \Omega(x)F^{\mu \nu}\Omega^{-1}(x)
\end{equation}
\textbf{Wilson lines} and \textbf{Wilson loops}.
\subsection{Quantization of Gauge Theory}
\subsection{Renormalization in Gauge Theory}
\end{document}