\documentclass[a4paper,11pt]{article}

%\usepackage{macros}
\usepackage{jheppub}

\usepackage[T1]{fontenc}
\usepackage{lmodern}
\usepackage{amsmath}
\usepackage{amsfonts}
\usepackage{amssymb}
\usepackage{amsthm}
\usepackage{mathrsfs}
\usepackage{physics}
\usepackage{graphicx}
\usepackage{subcaption}
\usepackage{enumitem}
\usepackage{framed,float}
\usepackage{booktabs}
\usepackage{array}
\usepackage{soul}
\usepackage{multirow}
\usepackage{comment}
\usepackage{braket}
\usepackage{todonotes}
\usepackage{youngtab}
\usepackage{tikz}
\usetikzlibrary{calc,arrows,decorations.markings}

%\usepackage{geometry}
\title{Generalized Symmetries}
\author{Qingjie Yuan}
\affiliation{Department of Physics and Astronomy, Uppsala University, \\Box 516, 75120 Uppsala, Sweden}
\emailAdd{qingjie.yuan.9461@uu.student.se}

\usepackage{color}

\begin{document}


\maketitle
\tableofcontents

\allowdisplaybreaks
%%%%%%%%
\section{Introduction}
\cite{Gaiotto:2014kfa}
\section{Higher-form symmetry}
\subsection{Symmetries as topological operators}
In classical field theory, a symmetry is a transformation of the fields that preserves the action. 
At quantum level, in the absence of anomalies, it is a transformation under which the correlation functions are invariant.\par
Considering the infinitisimal global transformation: $\phi\to \phi'=\phi+\epsilon_a \delta_a \phi$, $\delta S=S[\phi']-S[\phi]=0$ 


\subsection{Examples}
\subsubsection{Maxwell theory}
\subsubsection{SU(\textit{N}) gauge theory}

%%%%%%%%%%
\section{Spontaneous symmetry breaking}
\subsection{Normal symmetry}
\subsection{Higher-form symmetry}




\bibliographystyle{JHEP}
\bibliography{ref}
\end{document}