\documentclass[a4paper,11pt]{article}
%\usepackage{macros}


\usepackage[T1]{fontenc}
\usepackage{lmodern}
\usepackage{amsmath}
\usepackage{amsfonts}
\usepackage{amssymb}
\usepackage{amsthm}
\usepackage{mathrsfs}
\usepackage{physics}
\usepackage{graphicx}
\usepackage{caption}
\usepackage{subcaption}
\usepackage{enumitem}
\usepackage{framed,float}
\usepackage{booktabs}
\usepackage{array}
\usepackage{soul}
\usepackage{multirow}
\usepackage{comment}
\usepackage{braket}
\usepackage{todonotes}
\usepackage{youngtab}
\usepackage{tikz}
\usetikzlibrary{calc,arrows,decorations.markings}

\usepackage[left=0.5cm, right=0.5cm, top=1cm, bottom=1.5cm]{geometry}

\usepackage{color}

\newcommand{\rsh}[3]{[#1|#2|#3\rangle}
\newcommand{\lsh}[3]{\langle #1|#2|#3]}
\newcommand{\Xij}[2]{\braket{\eta #1}[#1 #2]\braket{#2 \eta}}
\newcommand{\etal}[1]{\braket{\eta #1}}
\newcommand{\etar}[1]{\braket{#1 \eta}}

\begin{document}
\section{General}
\begin{equation}
    A^{\text{ALSM}}(\tau_n)=(\pi i)^{n-2}=\sum_{\rho\in S_{n-2}}S[\rho(23\dots n-1)|\rho(23\dots n-1)]_1 m[1,\rho(2,\dots,n-1),n|\tau_n]
\end{equation}
\begin{equation}
    S[A,j|B,j,C]_i=(k_iB\cdot k_j)S[A|B,C]_i,\qquad S[\emptyset|\emptyset]\equiv 1
\end{equation}
\begin{equation*}
    k_i B\equiv k_i+k_{b_1}+\cdots+k_{b_{|B|}}
\end{equation*}
\begin{equation}
    m(P,n|Q,n)=s_P \phi_{P|Q}
\end{equation}
\begin{equation}
    \phi_{P|Q}=\frac{1}{s_P}\sum_{XY=P}\sum_{AB=Q}\bigg(\phi_{X|A}\phi_{Y|B}-(X\leftrightarrow Y)\bigg),\qquad \phi_{i|j}=\delta_{ij}
\end{equation}
\begin{equation*}
    s_P=k_P^2=(k_{P_1}+k_{P_2}+\cdots+k_{P_{|P|}})^2
\end{equation*}
An algorithm for Berends-Giele double-currents,
\begin{equation}
    \phi_i=1,\qquad \phi_P=\frac{1}{s_P}\sum_{XY=P}\phi_X \phi_Y, \qquad X,Y\neq \emptyset.
\end{equation}
\begin{equation}
    \begin{split}
    \phi_{12}=&\phi_{21}=\frac{1}{s_{12}}\\
    \phi_{123}=&\phi_{321}=\frac{1}{s_{123}}\bigg(\frac{1}{s_{12}}+\frac{1}{s_{23}}\bigg)\\
    \phi_{132}=&\phi_{231}=\frac{1}{s_{123}}\bigg(\frac{1}{s_{13}}+\frac{1}{s_{23}}\bigg)\\
    \phi_{213}=&\phi_{312}=\frac{1}{s_{123}}\bigg(\frac{1}{s_{12}}+\frac{1}{s_{13}}\bigg)
    \end{split}
\end{equation}


Shouten identity,
\begin{equation}
    \braket{ij}\braket{kl}+\braket{ik}\braket{lj}+\braket{il}\braket{jk}=0
\end{equation}
\begin{equation}
    [ij][kl]+[ik][lj]+[il][jk]=0
\end{equation}
\begin{equation}
    \braket{ij}[kl]+\braket{ik}[lj]+\braket{il}[jk]=0
\end{equation}

KLT relation,
\begin{equation}
    \mathcal{M}^{\text{tree}}_m=
    -i\sum_{\sigma,\rho\in S_{m-3}({2},...,m-2)}A_m^{\text{tree}}(1,\sigma,m-1,m)S[\sigma|\rho]\tilde{A}_m^{\text{tree}}(1,\rho,m,m-1)
\end{equation}
This $S[\sigma|\rho]$ is just the former $S[|]_1$.\par


%%%%%%%%%%
\section{NLSM Theory}
Four-point
\begin{equation}
    \begin{split}
    A^{\text{NLSM}}(\tau_4)=(\pi i)^2\bigg\{S[23|23]_1 m[1,2,3,4|\tau_4]+s[32|32]_1 m[1,3,2,4|\tau_4]\bigg\}
    \end{split}
\end{equation}
\begin{equation}
    s[23|23]_1=(s_{13}+s_{23})s_{12}=-s^2_{12}
\end{equation}
\begin{equation}
    s[32|32]_1=(s_{12}+s_{32})s_{13}=-s^2_{13}
\end{equation}
since $m(-|-)$ is cyclically symmetric in both, we need only consider,
\begin{equation}
    \tau_4=1234,1324,2134,2314,3124,3214
\end{equation}
(i) $\tau_4=1234$ or $3214$ (the result is the same for both cases since $\phi_{\sigma(123)|123}=\phi_{\sigma(123)|321}$ for any $\sigma\in S_3$ ),
\begin{equation}
    \begin{split}
    A^{\text{NLSM}}(1234)=A^{\text{NLSM}}(3214)=&(\pi i)^2\bigg\{-s^2_{12}s_{123} \phi_{123|123}-s^2_{12}s_{123} \phi_{132|123}\bigg\}\\
    =&(\pi i)^2\bigg\{-s^2_{12} \bigg(\frac{1}{s_{12}}+\frac{1}{s_{23}}\bigg)-s^2_{13}\frac{-1}{s_{23}}\bigg\}\\
    =&-(\pi i)^2 s_{13}
    \end{split}
\end{equation}
In the last line, $s_{ij}=k_i\cdot k_j$ is used.

(ii) $\tau_4=1324$ or $2314$,
\begin{equation}
    \begin{split}
    A^{\text{NLSM}}(1324)=A^{\text{NLSM}}(2314)=-(\pi i)^2 s_{12}
    \end{split}
\end{equation}

(iii) $\tau_4=2134$ or $3124$,
\begin{equation}
    \begin{split}
    A^{\text{NLSM}}(2134)=A^{\text{NLSM}}(3124)=-(\pi i)^2 s_{23}
    \end{split}
\end{equation}

Six-point [1608.02569,1304.3048]
\begin{equation}
    A^{\text{NLSM}}_6(1,2,3,4,5,6)=s_{12}-\frac{(s_{12}+s_{23})(s_{45}+s_{56})}{2s_{123}}+\text{cyc}(1,2,3,4,5,6)
\end{equation}

%%%%%%%%%%%%
\section{SDYM Theory}
\begin{equation}
    \begin{split}
    n^{sd}_{1|23|4}=&\braket{\eta r}^4\left(\prod_{i=1}^4 \frac{1}{\braket{\eta i}}\right)X_{1,2}X_{1+2,3}\\
    =&(common)\times\Big(X_{1,2}X_{1,3}+X_{1,2}X_{2,3}\Big)
    \end{split}
\end{equation}

\begin{equation}
    \begin{split}
    n^{sd}_{1|32|4}=&(common)\times X_{1,3}X_{1+3,2}\\
    =&(common)\times \Big(X_{1,2}X_{1,3}-X_{1,3}X_{2,3}\Big)
    \end{split}
\end{equation}

\begin{equation}
    \begin{split}
    A^{sd}_4(\underline{1},3,4,2)=&\frac{n^{sd}_{1|23|4}}{s_{12}}+\frac{n^{sd}_{1|32|4}}{s_{13}}\\
    =&(common)\times\bigg( -\frac{X_{1,2}X_{1,3}s_{23}}{s_{12}s_{23}}+X_{23}\frac{X_{1,2}s_{13}-X_{1,3}s_{12}}{s_{12}s_{13}}\bigg)\\
    =&\frac{(common)}{s_{12}s_{13}}\times\bigg(\braket{1\eta }[12]\braket{2\eta}\braket{\eta 1}[13]\braket{3\eta}\braket{23}[32]+
    \braket{\eta 2}[23]\braket{3\eta}\braket{\eta 1}[12][13]\Big(\braket{2\eta}\braket{31}-\braket{3\eta}\braket{21}\Big)\bigg)\\
    =&\frac{(common)}{s_{12}s_{13}}\times \braket{1\eta}\braket{2\eta}\braket{3\eta}[12][13][23]\bigg(\braket{\eta 1}\braket{32}+\braket{\eta 2}\braket{13}+\braket{\eta 3}\braket{21}\bigg)\\
    =0
    \end{split}
\end{equation}
In the last line, the Shouten identity is used.


5-point SDYM tree amplitude,
\begin{equation}
    \begin{split}
    A_5^{\text{SDYM}}(1,2,3,4,5)=&\frac{X_{1,2}X_{1+2,3}X_{1+2+3,4}}{s_{12}s_{123}}+(\text{cyc.})\\
    =&\frac{X_{1,2}X_{1+2,3}X_{4,5}}{s_{12}s_{45}}+(\text{cyc.})\\
    =&\frac{1}{2}\Bigg(\frac{X_{1,2}X_{1+2,3}X_{4,5}}{s_{12}s_{45}}-\frac{X_{1,2}X_{4+5,3}X_{4,5}}{s_{12}s_{45}}\Bigg)+(\text{cyc.})\\
    =&\frac{1}{2}\Bigg(\frac{X_{1,2}X_{1+2,3}X_{4,5}}{s_{12}s_{45}}-\frac{X_{4,5}X_{2+3,1}X_{2,3}}{s_{45}s_{23}}\Bigg)+(\text{cyc.})
    \end{split}
\end{equation}
The minus term in the last line is from the fourth term in $(\text{cyc.})$ of the third line.

\textbf{Lemma 1.}
\begin{equation}
    \frac{X_{i,j}X_{i+j,l}}{s_{ij}}-\frac{X_{j,l}X_{j+l,i}}{s_{jl}}=\frac{X_{i,j}X_{j,l}s_{ijl}}{s_{ij}s_{jl}}
\end{equation}
\textbf{Proof.}
\begin{equation}
    \begin{split}
        &\frac{X_{i,j}X_{i+j,l}}{s_{ij}}-\frac{X_{j,l}X_{j+l,i}}{s_{jl}}\\
        =&X_{i,j}X_{j,l}\frac{s_{ij}+s_{jl}}{s_{ij}s_{jl}}+X_{i,l}\frac{X_{i,j}s_{jl}+X_{j,l}s_{ij}}{s_{ij}s_{jl}}\\
        =&X_{i,j}X_{j,l}\frac{s_{ij}+s_{jl}}{s_{ij}s_{jl}}
        +\Xij{i}{l}\frac{\Xij{i}{j}\braket{jl}[lj]+\Xij{j}{l}\braket{ij}[ji]}{s_{ij}s_{jl}}\\
        =&X_{i,j}X_{j,l}\frac{s_{ij}+s_{jl}}{s_{ij}s_{jl}}
        +\frac{\Xij{i}{l}\braket{j \eta}\braket{\eta j }\braket{li}[jl][ij]}{s_{ij}s_{jl}}\\
        =&X_{i,j}X_{j,l}\frac{s_{ij}+s_{jl}}{s_{ij}s_{jl}}
        +\frac{X_{i,j}X_{j,l}s_{il}}{s_{ij}s_{jl}}\\
        =&\frac{X_{i,j}X_{j,l}s_{ijl}}{s_{ij}s_{jl}}.
    \end{split}
\end{equation}
By using Lemma 1, we can get,
\begin{equation}
    A_5^{\text{SDYM}}(1,2,3,4,5)=\frac{1}{2}\frac{X_{1,2}X_{2,3}X_{4,5}}{s_{12}s_{23}}+(\text{cyc.})
\end{equation}
Now I want to construct some cyclic symmetric term, which can be considered as a common factor.
\begin{equation}
    \begin{split}
        \frac{X_{1,2}X_{2,3}X_{4,5}}{s_{12}s_{23}}=
        \frac{\etal{1}\etal{2}\etal{3}\etal{4}\etal{5}}{\braket{12}\braket{23}\braket{34}\braket{45}\braket{51}}
        \times{s_{45}\etal{2}\braket{34}\braket{51}} 
    \end{split}
\end{equation}
So we have,
\begin{equation}
    A_5^{\text{SDYM}}(1,2,3,4,5)\propto {s_{45}\etal{2}\braket{34}\braket{51}}+(\text{cyc.})
\end{equation}
{\color{red}I have proved it is indeed vanishing by momentum conservation, I may type later.}

%%%%%%%%%%%%%%%%
\section{BI Theory}
[0808.2598]\\
Helicity selection-rule, the amplitudes is non-vanishing when number of positive helicity particle is the same as the number of the negative helicity particle. 
\begin{equation}
    M_4^{\text{BI}}(1+,2+,3-,4-)\propto [12]^2\braket{34}^2
\end{equation}

%%%%%%%%%%%
\section{Double-Copy Part}
$\mathcal{M}^{\text{BI}}=\mathcal{A}^{\text{YM}}\otimes \mathcal{A}^{\text{NLSM}}$.\\
Four-point tree-level KLT relation,
\begin{equation}
    M_4(1+,2+,3-,4-)=-i A_4^{\text{YM}}(1+,2+,3-,4-)S[2|2]A^{\text{NLSM}}_4(1,2,4,3)
\end{equation}
\begin{equation}
    LHS=[12]^2\braket{34}^2
\end{equation}
\begin{equation}
    \begin{split}
        RHS=&-\frac{\braket{34}^4}{\braket{12}\braket{23}\braket{34}\braket{41}}s_{12}s_{23}\\
        =&\frac{\braket{34}^3[12][23]}{\braket{14}}
    \end{split}
\end{equation}
we know 
\begin{equation}
    [23]\braket{34}=\rsh{2}{k_3}{4}=-\rsh{2}{k_1+k_2+k_4}{4}=-\rsh{2}{k_1}{4}=[12]\braket{14}
\end{equation}
Thus,
\begin{equation}
    RHS=\braket{34}^2[12]^2=LHS
\end{equation}

Five-point tree level MHV,
\begin{equation}
    \begin{split}
    0=M_5(++---)=
    \end{split}
\end{equation}

Six-point tree level,
\begin{equation}
    0=M_6(++----)=
\end{equation}

\begin{equation}
    \text{gravity}=\text{gauge}\otimes \text{gauge}
\end{equation} 
\begin{equation}
    \text{BI}=\text{YM}\otimes\text{NLSM}
\end{equation}

\[ \mathcolor{red}{c_i\leftrightarrow {n_i}}\]

\[f^{a_1 a_2 a_3}=-f^{a_2 a_1 a_3}\qquad F_{k_1 k_2 k_3}=\delta(k_1+k_2+k_3)\langle \eta |k_1k_2|\eta\rangle\]

\[F_{k_1 k_2 k_3}=-F_{k_2 k_1 k_3}\]
\end{document}