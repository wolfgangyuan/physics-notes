\documentclass[a4paper,11pt]{article}

\usepackage[T1]{fontenc}
\usepackage{lmodern}
\usepackage{amsmath}
\usepackage{amsfonts}
\usepackage{amssymb}
\usepackage{amsthm}
\usepackage{mathrsfs}
\usepackage{physics}
\usepackage{graphicx}
\usepackage{subcaption}
\usepackage{enumitem}
\usepackage{framed,float}
\usepackage{booktabs}
\usepackage{array}
\usepackage{soul}
\usepackage{multirow}
\usepackage{comment}
\usepackage{braket}
\usepackage{todonotes}
\usepackage{youngtab}
\usepackage{tikz}
\usetikzlibrary{calc,arrows,decorations.markings}
\title{TQFT}
\author{Qing-Jie Yuan}
\date{ }

\begin{document}
\maketitle
\tableofcontents

\section{Minimal category theory}

\section{Algebraic topology}
\section{Fiber bundles}
Why do we need fiber bundles? Let me share my understanding. 
We have a manifold $M$, and at each point on this manifold, there might be rich activity occurring. 
We use some new space $F$  to "accommodate" these phenomena. At first glance, it seems straightforward: 
just take the direct product of the manifold $M$ with space $F$. However, this only works at the set level. 
We cannot guarantee that it remains a direct product in a richer-structure sense.\par
Take a simple example: the cylinder and the Möbius strip. 
If you look point by point, each has an interval attached. 
Yet their overall topological structures are clearly distinct. 
Other examples like the hairy ball theorem and the Hopf fibration (readers can search for these themselves) illustrate the same point: 
just because each point on a manifold is "equipped" with the same space $F$, it doesn't mean the whole space can be expressed as a direct product $M\times F$.\par
So the question becomes: what mathematical language should we use to describe this?  
That's exactly where the definition of a fiber bundle comes in.

\subsection{Connections on fiber bundles}


\end{document}