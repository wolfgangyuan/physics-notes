\documentclass[a4paper,11pt]{article}

\usepackage[T1]{fontenc}
\usepackage{lmodern}
\usepackage{amsmath}
\usepackage{amsfonts}
\usepackage{amssymb}
\usepackage{amsthm}
\usepackage{mathrsfs}
\usepackage{physics}
\usepackage{graphicx}
\usepackage{subcaption}
\usepackage{enumitem}
\usepackage{framed,float}
\usepackage{booktabs}
\usepackage{array}
\usepackage{soul}
\usepackage{multirow}
\usepackage{comment}
\usepackage{braket}
\usepackage{todonotes}
\usepackage{youngtab}
\usepackage{slashed}
\usepackage{tikz}
\usetikzlibrary{calc,arrows,decorations.markings}
\title{Anomalies and Spontaneous Symmetry Breakings}
\author{Qing-Jie Yuan}
\date{ }

\begin{document}

\maketitle
\tableofcontents

\section{ABJ anomaly}
For a theory of massless Dirac fermions coupled to an electromagnetic gauge field in $d=3+1$ dimensions,
the action for the fermions is
\begin{equation}
    S=\int d^4 x\; i\bar{\psi}\slashed{D}\psi.
\end{equation} 
Here the gauge field is considered as a fixed backgroud field.
The symmetries for this action are,
\begin{equation}\label{ABJ-action}
    \psi\to e^{i \alpha}\psi
\end{equation}
with the corresponding current,
\begin{equation}
    j^\mu=\bar{\psi}\gamma^{\mu}\psi,
\end{equation}
and
\begin{equation}
    \psi\to e^{i \alpha \gamma^5}\psi
\end{equation}
with the corresponding current,
\begin{equation}
    j^{\mu}_A=\bar\psi \gamma^\mu \gamma^5 \psi.
\end{equation}
(\textbf{Aside.})
The action \eqref{ABJ-action} can be written as
\begin{equation}
    S=\int d^4 x\; i\bar{\psi}\slashed{\partial}\psi+j^{\mu}A_{\mu}.
\end{equation}
Thus if the action is invariant under the gauge transformation $A_\mu\to A_\mu+\partial_\mu \alpha$,
we have $\partial_\mu j^\mu=0$.\par
In classial theory, we should have $\partial_\mu j_A^{\mu}=0$. It is true, while in quantum theory,
\begin{equation}\label{chiral-anomaly}
    \partial_\mu j_A^{\mu}= \frac{e^2}{16\pi^2}\epsilon^{\mu \nu \rho \sigma}F_{\mu \nu}F_{\rho \sigma}.
\end{equation} 
This is ABJ or chiral anomaly.
\subsection{Deriving the chiral anomaly}

\section{Gauge anomalies}
Anomalies in gauge symmetries kill all physics completely.

\section{'t Hooft anomalies}
\begin{quote}
    A global symmetry with a 't Hooft anomaly remains a symmetry in the quantum
 theory. You only run into trouble if you couple the symmetry to a background gauge field, in which
 case the charge is no longer conserved. You run into real trouble if you try to couple
 the symmetry to a dynamical gauge field because then the 't Hooft anomaly becomes
 a gauge anomaly and the theory ceases to make sense. In other words, the 't Hooft
 anomaly is an obstruction to gauging a global symmetry.
\end{quote}



\end{document}